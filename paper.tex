\documentclass[10pt,letterpaper,oneside,draft]{article}

\usepackage{graphicx}
\usepackage{subfigure}

\usepackage[T1]{fontenc}
\usepackage[utf8]{inputenc}
\usepackage{authblk}

\usepackage[spanish,USenglish,UKenglish]{babel} 		% espanol, ingles
%\usepackage[latin1]{inputenc} 		% acentos sin codigo

\begin{document}

\selectlanguage{UKenglish}

\title{Privacy Composition Policies for Older Adults in Geriatric Centers}

%\author{Valeria Soto-Mendoza, Patricia Serrano-Alvarado, Emmanuel Desmontils and J. Antonio Garcia-Macias}

\author[1]{Valeria Soto-Mendoza\thanks{vsoto@cicese.edu.mx}}
\author[2]{Patricia Serrano-Alvarado\thanks{patricia.serranoalvarado@nantes-univ.fr}}
\author[2]{Emmanuel Desmontils\thanks{emmanuel@nantes-univ.fr}}
\author[1]{J. Antonio Garcia-Macias\thanks{jagm@cicese.mx}}
\author[3]{Jehu Hernandez\thanks{jehu.hdez@gmail.com}}

\affil[1]{Department of Computer Science, CICESE Research Center, Ensenada, Baja California, Mexico}
\affil[2]{LINA, University of Nantes, France}
\affil[3]{Softek, Ensenada, Mexico}

%\renewcommand\Authands{ and }

%\institute{CICESE Research Center, Ensenada, Baja California, Mexico \\ \email{vsoto@cicese.edu.mx, jagm@cicese.mx}}

\maketitle

\begin{abstract}
Policy composition has been studied 
\\
Contributions of our approach
\\
Provide a way to specify 
\\
- Purposes
\\
- duration of validity for a ``personalized'' policy
\\
- grantor/grantee, or licensor/licensee in a ``personalized'' policy
\\
- resources in a ``personalized'' policy
\\
Medical-scientific use case with a distributed architecture
\\
Take into account the point of view of data producers (patients). This is not frequently the case, in general, privacy terms are specified by institutions that will use/collect data and data-producers only sign (i.e., one-way policy terms) 
\\
Easy of use 
\\
More expressiveness
\\
- in a triple of the form <<s p o>> where s (subject) is a licence, p (property) are (among others) allows/prohibits/obliges and o (object) are terms (?) organized in purposes, operations (legal terms ?). See figure 
\\
- say clearly that all other terms not specified are prohibited

\end{abstract}

%\begin{keywords}
KEYWORDS: privacy, policy composition, older adults
%\end{keywords}

\section{Introduction}

\section{Related Work}

In the era of Web of Data, privacy is a relevant concern. The compatibility and composition of policies and licenses are topics 

\section{Our approach}

We propose the next architecture

We established that every user must manage their own data. For this reason, every entity should create privacy policies for their electronic resources as simply as possible, and in a machine-readable format.
\\
Figure X show the workflow of a policy (reference to emmanuel's diagram)
\\

The algorithm we propose is...


\section{Policies: an apply case of use}

\subsection{Usage scenarios}
% individual query
Every week a physician visit the residence to check the residents. He searches for the relevant data of each consulted resident, so he uses the system to get the information. The relevant information is  related with vitals signs (temperature, blood pressure, pulse, weight, etc.), meals, medicaments, and, if it is the case, the anomalies or additional comments from caregivers.

We define some policies in the context of older adults in geriatric centers: a patient?s policy, a scientist?s policy and a residence?s policy. We combine them according to the scenarios presented above. For the merge of two or more policies we use the next rules:

% table with operators

\section{Evaluation}
\subsection{Experimental setup}
\subsection{Dataset}
\subsection{Results}




\section{Conclusions and Future Work}

\begin{thebibliography}{10}

\bibitem{Rashidi2013}
Rashidi, P., Mihailidis, A.: {A survey on ambient-assisted living tools for
  older adults.} IEEE journal of biomedical and health informatics  17(3)

\end{thebibliography}


\end{document}